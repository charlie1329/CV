\documentclass[12pt]{article}
\usepackage{fullpage}
\usepackage{times}
\usepackage{hyperref}

\newcommand{\anytime}{\hyperlink{topicthree}{\textbf{Anytime Algorithms for Simultaneous Decision-Making and Verification}}}

\usepackage[style=ieee, backend=bibtex]{biblatex}
\addbibresource{../../references.bib}

\title{Research Statement\\ \large \emph{Robotic Systems with a Guaranteed Quality of Service under Uncertainty}}
\date{}
\author{}

\begin{document}
\maketitle
\thispagestyle{empty}
\pagenumbering{gobble}

\section*{Research Problem}

\vspace*{1ex}\noindent\textbf{Motivating Example.} Consider an autonomous fleet of drones and robotic quadrupeds tasked with finding uranium deposits in mines.
%
Mines are hostile environments with little light, dusty air, and falling debris. 
%
Robots have a limited battery, and must travel to a central `mothership' to recharge.
%
If a robot dies within a mine, it is difficult to retrieve, and in the worst case lost.
%
For safety and reliability we want to \emph{guarantee} that the probability of a robot's battery dying during exploration is very low.

Mine exploration is a challenging robotic problem that requires state-of-the-art solutions.
%
To efficiently explore the mine and identify uranium deposits we require automatic \emph{decision-making} techniques which determine how the robots should act during execution.
%
The results of decision-making will dictate where robots should map, and where to take rock samples.
%
To obtain robot battery level guarantees as described above, decision-making should occur simultaneously with \emph{formal verification}.
%
Formal verification methods rigorously analyse the evolution of the robots and environment to make quantitative or qualitative statements about robot performance.

Decision-making and verification methods often rely on a \emph{formal model} which describes how robot actions impact the environment, and how the environment evolves.
%
The success of these methods correlates strongly with model accuracy.
%
To construct accurate models, we must capture the sources of \emph{uncertainty} which affect the robots.
%
Uncertainty is highly prevalent during mine exploration.
%
For example, low light levels and dust may affect a robot's sensors, falling rocks may block parts of the mine, and robot communications may fail due to unexpected obstructions between robots.
%
Model construction is difficult, as robots have limited prior information about the mine.
%
Therefore, robots must reason over the uncertainty in their model to avoid potentially harmful decisions, such as mistakingly entering a high radiation area.

%If these models are inaccurate, our expectations of how the robots behave during verification and decision-making diverge from what we observe in the real world, limiting performance and weakening our guarantees. 
%
%
%Uncertainty affects the outcome and duration of robot actions, and a robot's ability to sense its surroundings.
%


\vspace*{1ex}\noindent\textbf{Research Goal.} The above problem is a use case for an EU Horizon project proposal I recently co-authored.
%
This use case is indicative of a more general open research problem which requires rich models of uncertainty, efficient decision-making, and formal verification.
%
My research goal is to tackle each of these sub-problems to develop \emph{robotic systems with a guaranteed quality of service under uncertainty}.
%
This requires robots that i) learn accurate models of uncertainty which are improved over their lifetime; and ii) exploit these models for efficient decision-making while satisfying formal guarantees.
%
This research is inherently cross-disciplinary, combining techniques from artificial intelligence, robotics, and formal verification.

\vspace*{1ex}\noindent\textbf{Existing Work.} Existing research has addressed components of the quality of service problem.
%
There are numerous robotic modelling techniques for capturing action outcome uncertainty, temporal uncertainty, partial observability, and the effects of robot interactions.
%
These techniques trade between model accuracy and the scalability of corresponding solution methods.
%
This balance often requires informed assumptions, such as localising where certain sources of uncertainty occur.
%
Advancements in modelling are not reflected in simultaneous verification and decision-making techniques, which are often limited to deterministic models or action outcome uncertainty.
%
This is often for scalability reasons, as richer forms of uncertainty are complex to model, and solution methods scale poorly as the model size increases.
%
I will address these issues through an \emph{holistic} approach to decision-making, verification, and modelling uncertainty.

\section*{Research Plan}

In my first five years as assistant professor, I will investigate the following:

\vspace*{1ex}\noindent\textbf{1) Simultaneous Robot Decision-Making and Verification under Complex Models of Uncertainty.} 
%
Existing decision-making and verification techniques are often limited to robot models which capture uncertainty over action outcomes.
%
This is not representative of real-world environments, where uncertainty also affects robot action durations and sensing.
%
To address this, I will establish novel techniques for decision-making and verification which handle more complex forms of uncertainty.
%
I developed expertise on this topic during my PhD, where I presented a suite of techniques for multi-robot modelling and decision-making under temporal uncertainty.
%
For example, I proposed novel temporal models which capture how robots affect each other's navigation performance.
%
I used these models to support multi-robot decision-making under congestion, where robots may take longer but less congested routes to reach their destination faster~\cite{street2020multi,street2021congestion}.
%
I also developed formal models which capture when and where robots will be requested for tasks.
%
I exploited these models for proactive decision-making, where robots can predict when and where they're needed, and arrive there early~\cite{street2024right}.
%
Both of these approaches utilise formal verification techniques to support decision-making, and provide an ideal entry point for research on simultaneous decision-making and verification.

\vspace*{1ex}\noindent\textbf{2) Improving Quality of Service Guarantees through Model Learning.}
%
Formal guarantees are only useful if the verification model is representative of the real environment.
%
In many problems, such as mine exploration, it is challenging for robots to obtain accurate models prior to deployment.
%
Robots should therefore learn and improve their model during execution, and use this to strengthen guarantees.
%
I've acquired model learning expertise from recent work on decision-making under uncertainty where the robot has no prior knowledge of its environment.
%
Here, the robot repeats a task periodically, and uses its observations during each run for model learning.
%
This gradual approach to model learning was shown to increase decision-making performance.
%
In my future research, I will apply this philosophy to different verification techniques to iteratively improve quality of service guarantees.

\vspace*{1ex}\noindent\hypertarget{topicthree}{\textbf{3) Anytime Algorithms for Simultaneous Decision-Making and Verification.}} 
%
As discussed above, simultaneous decision-making and verification techniques scale poorly with the size of robot models.
%
This becomes challenging when robots receive new tasks during execution, where the time for decision-making and verification is limited.
%
To mitigate this, I will investigate anytime algorithms for simultaneous decision-making and verification.
%
As these algorithms continue to run, decision-making performance and guarantees should improve.
%
I have a strong understanding of anytime decision-making methods, which have appeared frequently throughout my research.
%
During my PhD, I applied anytime algorithms for multi-robot decision-making under rich models of congestion~\cite{street2020multi,street2021congestion}.
%
In recent work, I applied anytime methods for decision-making problems where the robot could only partially sense its environment.
%
Anytime decision-making algorithms share a similar philosophy with statistical verification methods.
%
The first step of my research will be to investigate how these techniques can be combined in a principled way for simultaneous decision-making and verification.

\vspace*{1ex}\noindent\textbf{Research Philosophy.} I believe in a \emph{practical approach} to robotics research.
%
The benefits of any robotic technique are not fully understood until they have been deployed on hardware to solve a real-world problem.
%
This brings many practical challenges, but is essential for effective dissemination across the robotics community.
%
Moreover, hardware deployments often highlight interesting, undiscovered problems which feed back into the research process.

\section*{Dissemination \& Collaborations}

\vspace*{1ex}\noindent\textbf{Dissemination.} I have a strong publication record in top venues such as the \emph{IEEE Transactions on Robotics (T-RO)}, the \emph{Journal for Artificial Intelligence Research (JAIR)}, and the \emph{International Conference on Autonomous Agents and Multi-Agent Systems (AAMAS)}.
%
To highlight the current challenges in my field, I ran a half day tutorial on multi-robot planning under uncertainty at AAMAS 2023.
%
Here, I covered the sources of uncertainty which affect multi-robot systems, and discussed how researchers can design decision-making techniques to make robots robust to these effects.
%
This was supported with a survey article published in \emph{Springer's Current Robotics Reports}~\cite{street2023formal}.\looseness=-1

\vspace*{1ex}\noindent\textbf{Collaborations.} The strength of my PhD research spawned two external collaborations.
%
I worked with the University of Lincoln on the First Fleet project to deploy multi-robot decision-making techniques I developed onto agricultural robots in fruit fields.
%
Fruit fields constrain robot movement, introducing interesting challenges for the guaranteed quality of service problem, as one poor decision may cause a robot to traverse large parts of the field.
%
I also led a collaboration with Accenture Labs to apply my novel multi-agent modelling formalisms to evaluate human-robot teams in warehouses~\cite{street2022context,street2023analysing}.
%
Warehouses often contain tens to hundreds of agents.
%
This necessitates the use of sampling-based verification techniques which will be fundamental to my work on \anytime. 


I am a technical work package lead in EU Horizon project CONVINCE, where I lead research on task and motion planning in dynamic environments.
%
CONVINCE is developing a fully verifiable toolchain for robotic systems, and is made up of partners from academia, industry, and the public sector.
%
My expertise in modelling uncertainty, decision-making, and verification allows for close collaboration within the consortium, especially with the verification team.
%
This will be reflected in the project's first software release and upcoming collaborative works.

The school of computer science will provide me with opportunities to foster new research collaborations.
%
There are academics within the school whose research interests overlap with mine.
%
For example, Dr Mirco Giacobbe's research lies between formal verification and artificial intelligence, and Dr Leonardo Stella investigates multi-agent decision-making through reinforcement learning.
%
I have spoken to Mirco and Leonardo about my work, and recently co-authored an application to the Google Research Scholar Program with Leonardo on applying his multi-agent reinforcment learning methods to robotics.


\section*{Summary}

My research has produced rich models of uncertainty and decision-making techniques which exploit them.
%
Formal verification and model checking have been at the core of my work but never at the forefront.
%
As outlined in this statement, I will address this by developing novel frameworks for guaranteed quality of service which can be deployed on physical robotic systems for safety and reliability.
%
This research brings numerous challenges, and the inter-disciplinary expertise in the school of computer science will provide an exicting environment to address them in.

\printbibliography

\end{document}