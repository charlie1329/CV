\documentclass[11pt]{article}
\usepackage{fullpage}
\usepackage{times}

\title{Research Statement\\ \large \emph{Robotic Systems with a Guaranteed Quality of Service under Uncertainty}}
\date{}
\author{}

\begin{document}
\maketitle
\thispagestyle{empty}
\pagenumbering{gobble}

As robots become more prevalent in healthcare, manufacturing, and on our roads, there is a growing need for safe and reliable autonomy.
%
Safety and reliability can be quantified through \emph{formal guarantees}, i.e. specifications the robot will always satisfy.
%
For example, an autonomous vehicle may guarantee that road rules are violated in less than $1\%$ of its operation time.
%
To obtain robust guarantees, robots require accurate \emph{models} of the environment, and how its actions affect it.
%
Existing robot deployments often assume limited, controlled environments which admit simplistic, deterministic models.
%
However, real world environments are stochastic, with numerous sources of uncertainty which affects the outcome of robot actions, their duration, and the robot's ability to observe its surroundings.
%
Identifying relevant sources of uncertainty for a problem is challenging, and introduces trade offs between accuracy and the scalability of formally verifying the system.


% NEW PARAGRAPH

Formal guarantees should coincide with efficient robot behaviour.
%
If a robot performs poorly, guarantees have little use, as there is little benefit in deploying the robot.
%
Therefore, we require decision-making techniques which synthesise robot behaviour that satisfies a given guarantee or specification.
%
Similar to verification, decision-making techniques rely on accurate models.
%
If a model is inaccurate, our expectations of robot behaviour during decision-making diverge from what we observe during execution, limiting performance.

% NEW PARAGRAPH

Any decision-making or verification model will contain inaccuracies.
%
Therefore, robots should be embodied with techniques to reason over the epistemic uncertainty in their models: how much do they know, and how certain are they of it?
%
With this, robots may choose to explore to improve their model, and guarantees may provide bounds given model uncertainty.

% BIG SEARCH AND RESCUE EXAMPLE

As an example, consider a heterogeneous team of rugged wheeled robots and drones in a search and rescue scenario.
%
Here, drones must identify human survivors who are then retrieved by the wheeled robots.
%
This is a safety critical scenario with obvious desired properties, e.g. $99\%$ of survivors are rescued unharmed.
%
This domain has numerous sources of uncertainty.
%
Smoke blown by the wind may restrict a drone's visibility.
%
The spread of fire or rubble may affect a wheeled robot's ability to navigate.
%
Broken power lines may affect robot communication.
%
Finally, the robots affect each other, e.g. if two wheeled robots navigate near each other, they may slow each other down.
%
To solve this problem, we require models which capture spatial uncertainty, temporal uncertainty, and partial observability.
%
The complexity of these models requires novel coordination solutions; simultaneous verification and decision-making is significantly more complex than standard decision-making.

SOMETHING ABOUT MODEL UNCERTAINTY IN THIS EXAMPLE

% SUMMARISE THE PROBLEM THAT WE WANT TO ACHIEVE

My research goal is to develop robotic systems which provide a guaranteed quality of service under uncertainty.
%
This requires robots which can i) learn accurate models of their environment and improve them over time; ii) use these models to synthesise efficient behaviour; and iii) provide formal guarantees over this behaviour.
%
This research is inherently cross-disciplinary, combining techniques from the AI, robotics, and formal verification communities.
%


Research within the AI, robotics, and formal methods communities have begun to address components of the quality of service problem.
%
There have been numerous strands of work on robotic modelling which capture different forms of uncertainty.
%
These models inherently trade off between how accurately we capture uncertainty, and how tractably these models can be solved and verified.
%
Treading this line often involves making smart simplifying assumptions about the problem at hand, such as limiting where certain sources of uncertainty may occur.
%
There is a gap between state-of-the-art modelling techniques, and the current ability of simultaneous verification and decision-making techniques, which are often limited to only consider action outcome uncertainty.
%
This is partially due to a lack of scalability: richer forms of uncertainty are more complex to model, and combined decision-making and verification algorithms scale poorly with model complexity.
%
Moreover, all of the aforementioned problems are amplified as we start to consider fleets of robots and  model uncertainty.

My interest in quality of service guarantees for robotic systems began during my PhD.
%
My thesis presented multiple techniques for multi-robot coordination under temporal uncertainty, with a focus on modelling and decision-making.
%
In one piece of work, presented in the IEEE transactions on robotics\footnote{Street, C., Pütz, S., Mühlig, M., Hawes, N. and Lacerda, B., 2022. Congestion-aware policy synthesis for multirobot systems. IEEE Transactions on Robotics, 38(1), pp.262-280.} I constructed temporal models which capture how robots affect each others' navigation performance when they operate in the same area simultaneously.
%
This allowed us to plan for multiple robots under congestion, i.e. one robot may take a longer but less congested route as it is lkely to reach its goal quicker.
%
In work presented in the journal of artificial intelligence research\footnote{Street, C., Lacerda, B., Mühlig, M. and Hawes, N., 2024. Right Place, Right Time: Proactive Multi-Robot Task Allocation Under Spatiotemporal Uncertainty. Journal of Artificial Intelligence Research, 79, pp.137-171.}, I developed temporal models which capture when and where robots will be required for tasks.
%
This was used to admit proactive decision-making, where robots predict when and where they are needed, and arrive there early.
%
These kinds of modelling and decision-making techniques are crucial for developing performant robots with a guaranteed quality of service.
%
Though the above techniques do not provide guarantees, model checking techniques are core to their success.
%
By model checking our temporal models, we can provide accurate predictions to support decision-making.
%
Following my PhD, I summarised the state-of-the-art in multi-robot modelling under uncertainty for the community into a survey article in Springer's current robotics reports\footnote{Street, C., Mansouri, M. and Lacerda, B., 2023. Formal Modelling for Multi-Robot Systems Under Uncertainty. Current Robotics Reports, 4(3), pp.55-64.}, and gave a half day tutorial in the international conference on autonomous agents and multi-agent systems.
% Talk about my PhD work and how it's been disseminated (tutorial here!)

% Talk about applications of this work (Lincoln, Accenture)

% Something about CONVINCE

% Where I go next

% Having things working on real robot systems

% Why UoB?



\end{document}