\documentclass[11pt]{article}
\usepackage{fullpage}
\usepackage{times}

\title{Research Statement\\ \large \emph{Robotic Systems with a Guaranteed Quality of Service under Uncertainty}}
\date{}
\author{}

\begin{document}
\maketitle
\thispagestyle{empty}
\pagenumbering{gobble}


As robotic systems become prevalent in healthcare, manufacturing, and on our roads, there is a growing need for safe and reliable autonomy.
%
For this, we desire \emph{formal guarantees} over the behaviour of our system.
%
For example, we may want to ensure that an autonomous vehicle only breaks road rules when there is danger to life.
%
Formal guarantees are often specified by the system designer in a temporal logic.
%
Robot behaviour should then be synthesised automatically to satisfy this specification.
%
This requires tight coupling between formal verification, and robot coordination and decision-making.
%
To obtain accurate guarantees and efficient behaviour, robots require models which capture the sources of \emph{uncertainty} that affect their behaviour in real-world environments.
%
Uncertainty affects the outcome and duration of robot actions, and a robot's ability to sense its surroundings.
%
If robot models are inaccurate, our expectations of behaviour during verification and coordination diverge from what we observe during execution, limiting task performance and weakening guarantees.
%
However, model inaccuracies are unavoidable due to limited data etc.
%
Therefore, robots must reason over the \emph{epistemic uncertainty} in their models to bound guarantees, and acquire knowledge through physical interactions with the environment.


\iffalse
%
To obtain robust guarantees, robots require accurate \emph{models} of the environment, and how its actions affect it.
%
Existing robot deployments often assume limited, controlled environments which admit simplistic, deterministic models.
%
However, real world environments are stochastic, with numerous sources of uncertainty which affects the outcome of robot actions, their duration, and the robot's ability to observe its surroundings.
%
Identifying relevant sources of uncertainty for a problem is challenging, and introduces trade offs between accuracy and the scalability of formally verifying the system.


% NEW PARAGRAPH

Formal guarantees should coincide with efficient robot behaviour.
%
If a robot performs poorly, guarantees have little use, as there is little benefit in deploying the robot.
%
Therefore, we require decision-making techniques which synthesise robot behaviour that satisfies a given guarantee or specification.
%
Similar to verification, decision-making techniques rely on accurate models.
%
If a model is inaccurate, our expectations of robot behaviour during decision-making diverge from what we observe during execution, limiting performance.


% NEW PARAGRAPH

Any decision-making or verification model will contain inaccuracies.
%
Therefore, robots should be embodied with techniques to reason over the epistemic uncertainty in their models: how much do they know, and how certain are they of it?
%
With this, robots may choose to explore to improve their model, and guarantees may provide bounds given model uncertainty.

\fi

% BIG SEARCH AND RESCUE EXAMPLE

As an example, consider a heterogeneous team of rugged wheeled robots and drones in a search and rescue scenario.
%
Here, drones must identify human survivors who are then retrieved by the wheeled robots.
%
This is a safety critical scenario with obvious desired properties, e.g. $99\%$ of survivors are rescued unharmed.
%
This domain has numerous sources of uncertainty.
%
Smoke blown by the wind may restrict a drone's visibility.
%
The spread of fire or rubble may affect a wheeled robot's ability to navigate.
%
Broken power lines may affect robot communication.
%
Finally, the robots affect each other, e.g. if two wheeled robots navigate near each other, they may slow each other down.
%
To solve this problem, we require models which capture spatial uncertainty, temporal uncertainty, and partial observability.
%
The complexity of these models requires novel coordination solutions; simultaneous verification and decision-making is significantly more complex than standard decision-making.

SOMETHING ABOUT MODEL UNCERTAINTY IN THIS EXAMPLE

% SUMMARISE THE PROBLEM THAT WE WANT TO ACHIEVE

My research goal is to develop robotic systems which provide a guaranteed quality of service under uncertainty.
%
This requires robots which can i) learn accurate models of their environment and improve them over time; ii) use these models to synthesise efficient behaviour; and iii) provide formal guarantees over this behaviour.
%
This research is inherently cross-disciplinary, combining techniques from the AI, robotics, and formal verification communities.
%


Research within the AI, robotics, and formal methods communities have begun to address components of the quality of service problem.
%
There have been numerous strands of work on robotic modelling which capture different forms of uncertainty.
%
These models inherently trade off between how accurately we capture uncertainty, and how tractably these models can be solved and verified.
%
Treading this line often involves making smart simplifying assumptions about the problem at hand, such as limiting where certain sources of uncertainty may occur.
%
There is a gap between state-of-the-art modelling techniques, and the current ability of simultaneous verification and decision-making techniques, which are often limited to only consider action outcome uncertainty.
%
This is partially due to a lack of scalability: richer forms of uncertainty are more complex to model, and combined decision-making and verification algorithms scale poorly with model complexity.
%
Moreover, all of the aforementioned problems are amplified as we start to consider fleets of robots and  model uncertainty.

My interest in quality of service guarantees for robotic systems began during my PhD.
%
My thesis presented multiple techniques for multi-robot coordination under temporal uncertainty, with a focus on modelling and decision-making.
%
In one piece of work, presented in the IEEE transactions on robotics\footnote{Street, C., Pütz, S., Mühlig, M., Hawes, N. and Lacerda, B., 2022. Congestion-aware policy synthesis for multirobot systems. IEEE Transactions on Robotics, 38(1), pp.262-280.} I constructed temporal models which capture how robots affect each others' navigation performance when they operate in the same area simultaneously.
%
This allowed us to plan for multiple robots under congestion, i.e. one robot may take a longer but less congested route as it is lkely to reach its goal quicker.
%
In work presented in the journal of artificial intelligence research\footnote{Street, C., Lacerda, B., Mühlig, M. and Hawes, N., 2024. Right Place, Right Time: Proactive Multi-Robot Task Allocation Under Spatiotemporal Uncertainty. Journal of Artificial Intelligence Research, 79, pp.137-171.}, I developed temporal models which capture when and where robots will be required for tasks.
%
This was used to admit proactive decision-making, where robots predict when and where they are needed, and arrive there early.
%
These kinds of modelling and decision-making techniques are crucial for developing performant robots with a guaranteed quality of service.
%
Though the above techniques do not provide guarantees, model checking techniques are core to their success.
%
By model checking our temporal models, we can provide accurate predictions to support decision-making.
%
Following my PhD, I summarised the state-of-the-art in multi-robot modelling under uncertainty for the community into a survey article in Springer's current robotics reports\footnote{Street, C., Mansouri, M. and Lacerda, B., 2023. Formal Modelling for Multi-Robot Systems Under Uncertainty. Current Robotics Reports, 4(3), pp.55-64.}, and gave a half day tutorial in the international conference on autonomous agents and multi-agent systems.

% Talk about applications of this work (Lincoln, Accenture)

The practical benefits of my PhD work have been demonstrated through external collaborations.
%
I've worked with the university of Lincoln\footnote{https://lcas.lincoln.ac.uk/wp/} on deploying my congestion-aware multi-robot planner onto agricultural robots aiding human fruit pickers.
%
This proposes interesting challenges for robot decision-making, as robots are constrained to narrow aisles in the fields.
%
I've also worked with Accenture Lab\footnote{https://www.accenture.com/gb-en/about/accenture-labs-index} on applying multi-agent modelling techniques developed during my PhD\footnote{Street, C., Lacerda, B., Staniaszek, M., Mühlig, M. and Hawes, N., 2022. Context-aware modelling for multi-robot systems under uncertainty.} on order picking systems in warehouses.
%
The aim was to utilise my models to evaluate the effects of congestion between robots and humans on KPIs such as throughput in warehouses.
%
My models capture congestion more precisely than existing approaches, and revealed key insights into when robotic systems are useful for businesses, and how many should be purchased\footnote{Street, C., Jujjavarapu, S.S., Chen, M.N.A., Paul, S. and Hawes, N., 2023, July. Analysing the Effects of Congestion on Hybrid Order Picking Systems Using a Discrete-Event Simulator. In International Conference on Intelligent Autonomous Systems (pp. 393-404). Cham: Springer Nature Switzerland.}.

% Something about CONVINCE

My research goals have broadened during my recent work on EU horizon project CONVINCE\footnote{https://convince-project.eu/}.
%
The aim of CONVINCE is to develop a fully verifiable toolchain for robotic systems.
%
My role within CONVINCE involves robot planning in dynamic and uncertain environments
%
In one line of research, I explored techniques for robots to efficiently cover environments given no prior knowledge of its dynamics.
%
Here, the robot should learn the dynamics over its lifetime, and reason over how much it knows about its environment.
%
An alternate line of research investigates behaviour trees, a popular software formalism for defining robot behaviour.
%
These are often designed in an ad-hoc manner, without formal guarantees or any consideration of uncertainty.
%
I have developed planning techniques which refine existing behaviour trees to attain robustness under uncertainty.
%
Such techniques also provide the groundwork for introducing model checking methods to achieve formal guarantees.

% Having things working on real robot systems

I believe strongly in the important of testing and evaluating robotic technologies on physical systems to solve real-world problems.
%
This has been demonstrated through my work with external collaborators, and time spent leading Oxford's RoboCup team.
%
I argue that the benefit of any methodology for cyber-physical systems isn't truly understood until it has been deployed on hardware.
%
Real-world deployments provide a unique set of problems which often feedback into the research process.
%
This focus on hardware will form a core philosophy of my future research, and the resources within the school of computer science will enable this.

% Where I go next & Why UoB?


My existing work has opened multiple future strands of research.
%
How can we retain richer forms of uncertainty in our models while controlling their size?
%
How can we develop more scalable decision-making techniques which exploit rich models to synthesise efficient and robust robot behaviour?
%
Further to this, how can we simultaneously verify robot behaviour during decision-making on large, complex models in a tractable way?
%
How can we incorporate epistemic uncertainty into simultaneous decision-making and verification methods?
%
These questions require varied solutions, and the inter-disciplinary expertise in the school of computer science will provide a great environment to foster new collaborations to address these challenges.
\end{document}