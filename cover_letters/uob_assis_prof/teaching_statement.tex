\documentclass[12pt]{article}
\usepackage{fullpage}
\usepackage{times}
\usepackage{url}

\title{Teaching Statement}
\date{}
\author{}

\begin{document}
\maketitle
\thispagestyle{empty}
\pagenumbering{gobble}


Teaching is crucial to academic life, and should never be left as an afterthought.
%
I have learned this from my time as a teacher and student.
%
A researcher's ideas will have little impact if they cannot explain them clearly and succinctly to experts and non-experts alike.
%
Teaching provides a prime opportunity to develop communication and presentation skills.
%
One of my goals as a teacher is to elicit challenging questions from students.
%
This demonstrates understanding and a critical engagement with the material that is important for undergraduates and postgraduates.
%
For postgraduate supervision, I aim to train researchers who are receptive to different ideas while being able to defend their own.
%
In this statement, I will outline my approaches to teaching and supervision which achieve these goals, and the experience which has shaped them.
%
I am eager to continue teaching and develop new and engaging ways of communicating ideas to students and colleagues.\looseness=-1

\iffalse
Teaching is essential to academic life and should never be dismissed as an afterthought.
%
Academics cannot succeed without the ability to clearly, succinctly explain their ideas.
%
In particular, they should be able to communicate with non-experts; teaching provides the perfect opportunity to develop this skill.
%
%
Teaching should be engaging and stimulating for students.

I'm early in my teaching career, but look forward to developing interesting and entertaining teaching methods through an active dialogue with students.

From my experience teaching, and as a student, I've found that grounding content into practical real-world examples, providing hands on work, and covering state-of-the-art methods and ideas are key ways to achieve this.
%
An indicator of a teacher's success is the questions they receive from students.
% 
If a teacher receives novel, challenging questions, the students have understood the material and are engaging with it critically.
%
This should be the ultimate goal of teaching, and is what I aim for.
%
This is particularly applicable to postgraduates, who we should teach to view the world with a critical eye and develop original ideas.
\fi


In the last two years I have given lectures and tutorials on multi-robot planning under uncertainty for the Advanced Robotics course.
%
I also helped design the 2023-2024 exam.
%
I gave an extended version of this lecture as a half-day tutorial at the 2023 International Conference on Autonomous Agents and Multi-Agent Systems (AAMAS).
%
The tutorial received positive feedback and had high audience retention.
%
I presented this material from a designer's perspective, i.e. given a real-world problem, which techniques are most appropriate and why?
%
The aim was to get students thinking about these methods critically and understanding the trade-offs between them.
%
During my PhD I led the University of Oxford's RoboCup team\footnote{\url{https://ori.ox.ac.uk/student-teams/team-orion/}}.
%
This required recruiting, training, and transferring knowledge with undergraduate students.
%
These students often had no prior robotics experience, and in some cases little to no programming experience.
%
I had to teach students under tight deadlines in order to make meaningful contributions to annual competitions. 
%
I achieved this through lab sessions, tutorials, and deferring more specialised training into separate sub-teams.
%
%Managing and training the team developed leadership and communication skills which will transfer to teaching and leading modules.
%
I have also worked as a TA for a week-long robotics crash course aimed at first-year CDT students.
%
I developed practical assignments which balanced between challenge and feasibility within the limited course duration.
%
The assignments had to fit together for a real robot competition run at the end of the week.
%
This course developed my ability to convey complex topics in a short time.
%
This experience gives me confidence to lead modules within the school of computer science.

A key skill for teachers is knowing how to keep students engaged.
%
I achieve this through a practical, example-focused approach to teaching.
%
In my experience theoretical concepts are easier to digest if applied to realistic examples, or demonstrated through practical work.
%
I apply this same philosophy when designing assessments or examinations.
%
As well as improving engagement, this helps undergraduate students adopt a problem-focused mindset which prepares them for their future careers.
%
I also believe my enthusiastic, energetic teaching style further helps keep students engaged.

I've supervised many undergraduate, masters, and internship projects, and have designed multiple of these.
%
I've learned to develop contained ideas which remain interesting and intellectually stimulating for the student.
%
I approach undergraduate supervision similarly to PhD supervision, which I describe below.
%
I treat supervision as a dialogue with the student rather than just instructing them what to do.
%
With this, I want to help students think critically about their work, and prepare them for the challenges they'll face in industry.


%Though early in my career, I am beginning to develop my supervision style based on my experience as PhD student and as a supervisor.
%
I currently co-supervise two PhD students, one at UoB, and one at the Italian Institute of Technology (IIT).
%
Though supervision style is always tailored to the student, I believe in a few core principles.
%
It is important to help PhD students structure their PhD to keep a sense of progress.
%
This is essential in maintaining student morale and wellbeing.
%
For this, I often suggest a practical, incremental approach to research.
%
This does not refer to research output, but in how to tackle research problems.
%
PhD students, myself included, often try to find the perfect solution in their first attempt.
%
This is rarely possible, or at least not immediately visible.
%
A more incrememental approach helps a student quickly learn and understand the problem, even if their final solution looks different.
%
This is particularly useful at the beginning of a PhD.
%
Alongside understanding the relevant litature, a student can start thinking about solutions using state-of-the-art methods as a jumping off point, for example.
%
Aside from research, a PhD is a chance for students to develop their soft skills such as communication.
%
To support this, I want to encourage my students to take development opportunities such as summer schools, doctorial consortia, and conferences.

%PhD supervision should also be a two-way process, and a supervisor should always be receptive to learning from their students.
\end{document}