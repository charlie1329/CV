\documentclass[12pt]{article}
\usepackage{fullpage}
\usepackage{times}
\usepackage{url}

\title{Teaching Statement}
\date{}
\author{}

\begin{document}
\maketitle
\thispagestyle{empty}
\pagenumbering{gobble}


Teaching is essential to academic life and should never be dismissed as an afterthought.
%
Academics cannot succeed without the ability to clearly, succinctly explain their ideas.
%
In particular, they should be able to communicate with non-experts; teaching provides the perfect opportunity to develop this skill.
%
Teaching should be engaging and stimulating for students.
%
From my experience teaching, and as a student, I've found that grounding content into practical real-world examples, providing hands on work, and covering state-of-the-art methods and ideas are key ways to achieve this.
%
An indicator of a teacher's success is the questions they receive from students.
% 
If a teacher receives novel, challenging questions, the students have understood the material and are engaging with it critically.
%
This should be the ultimate goal of teaching, and is what I aim for.
%
This is particularly applicable to postgraduates, who we should teach to view the world with a critical eye and develop original ideas.
%
I'm early in my teaching career, but look forward to developing interesting and entertaining teaching methods through an active dialogue with students.
% I enjoy doing this (but better)

In the past two academic years, I have given lectures and tutorials on multi-robot planning under uncertainty for the Advanced Robotics course.
%
In the 2023-2024 academic year, I also designed a 20 mark exam question for the course based on my lecture content.
%
I gave an extended version of this lecture as a half-day tutorial at the 2023 International Conference on Autonomous Agents and Multi-Agent Systems.
%
In these lectures I presented the material from a designer's perspective: given a real world problem a student may receive in academia or industry, which techniques are most appropriate and why?
%
The aim here was not to overload students with theory, but to get them to engage with the trade offs between different techniques, and when some may be more appropriate.
%
During my PhD I led the University of Oxford's RoboCup team\footnote{\url{https://ori.ox.ac.uk/student-teams/team-orion/}} for two years.
%
Though not directly teaching, this required recruiting, training, and transferring knowledge with undergraduate students.
%
These students often had no prior robotics experience, and in some cases, little programming experience.
%
Students had to learn the basics quickly to make meaningful contributions to annual competitions.
%
This was achieved through lab sessions, tutorials, and by splitting more specific training into separate sub teams.
%
Managing and training the team developed leadership and communication skills which will transfer to teaching and leading modules.
%
In 2020, I worked as a TA for a week-long robotics course aimed at first-year CDT students.
%
Here I helped develop practical assignments which balanced between challenge and feasibility within the limited course duration.
%
These assignments had to fit together to form a final real robot trial the students would run at the end of the week.
%
This crash course helped me understand how to approach complex topics in a way that is easily digestible for students with a limited time to understand it.
%
With my prior experience, I'm confident in my ability to independently lead modules and design exams/continuous assessments etc.

In undergraduate projects, discuss about designing projects, i.e. Rishi

Though early in my career, I am beginning to develop my supervision style based on my experience as PhD student and as a supervisor.
%
I'm currently on the supervision team for two students, one at UoB, and on at the Italian Institute of Technology (IIT).
%
Though supervision style will of course be tailored to the student, there are core principles I find to be helpful.
%
It is important to help students in structuring their PhD to allow a sense of progress.
%
Progress is essential to maintaining morale and wellbeing during a PhD, which is an endurance challenge.
%
I find a practical, incremental approach to research can help with this.
%
This does not necessarily refer to research output, but in how to tackle and solve research problems.
%
There is a tendency for PhD students (myself included) to try and find a perfect solution straight away.
%
This is rarely possible, or at least not immediately visible.
%
Instead, trying to tackle a problem incrementally helps a student learn and understand the problem, even if a different final approach is used.
%
This can be particularly at the start of a PhD, particularly in more practical fields like robotics.
%
Understanding the relevant literature is an essential starting point, but it can really help development to be working on something practical in the mean time, using state-of-the-art methods as a jumping off point, for example.

\end{document}