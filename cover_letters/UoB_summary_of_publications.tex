\documentclass[12pt]{article}
\usepackage{fullpage}

% Publication list
\usepackage[sorting=none, backend=bibtex, doi=false, isbn=false]{biblatex}
\addbibresource{../references.bib}
\preto\fullcite{\AtNextCite{\defcounter{maxnames}{99}}}

\title{Summary of Publications}
\date{}
\author{Charlie Street}

\begin{document}
\maketitle
\thispagestyle{empty}
\begin{enumerate}
\item[\cite{street2022context}] \fullcite{street2022context}.


Formal task-level models of multi-robot behaviour are fundamental to planning, simulation, and model checking techniques.
%
However, existing multi-robot models such as multi-agent Markov decision processes (MDPs) are invalidated by strong assumptions which fail to capture execution-time behaviour and limit their predictive power, such as simplistic duration models or synchronisation constraints.
%
In this paper we propose a novel \emph{multi-robot Markov automaton} (MA) formulation which models asynchronous multi-robot execution in continuous time.
%
MA extend MDPs by separating instantaneous decision making from stochastic action durations.
%
We capture robot dynamics, i.e. the behaviour of the controller and motion planner, using \emph{phase-type distributions} (PTDs) over action durations.
%
Further, the duration and outcome of robot actions is dependent on the precise spatiotemporal situation in which they are executed, which we refer to as the \emph{context}.
%
Contexts allow us to model the effects of environmental dynamics and robot interactions on robot action execution, where we fit a PTD for each context an action is executed in.
%
Within CONVINCE, contexts can capture human presence within a museum, which affects a robotic guide's navigation performance, or the state of a living room a robotic vacuum cleaner is operating in.
%
We also present a scalable discrete-event simulator which yields realistic statistics over execution-time robot behaviour by sampling through the multi-robot MA.
% TODO: Add avoiding computational overhead in somewhere?
%
Within CONVINCE, our simulator could be used alongside statistical model checking techniques to verify robot behaviour, or to rapidly evaluate contingency plans during execution.
%
Empirically, we demonstrate that our MA formulation can accurately predict the task-level behaviour of a multi-robot system simulated in Gazebo.

\item[\cite{street2020multi}] \fullcite{street2020multi}.

Multi-robot systems must be able to maintain performance when robots get delayed during execution.
%
For mobile robots, one source of delays is \emph{congestion}.
%
Congestion occurs when robots deployed in shared physical spaces interact, as robots present in the same area simultaneously must manoeuvre to avoid each other.
%
Congestion can adversely affect navigation performance, and increase the duration of navigation actions.
%
In this paper, we present a multi-robot planning framework which utilises learnt probabilistic models of how congestion affects navigation duration. 
%
These duration models are realised as PTDs for each action and congestion level, similar to the multi-robot MA discussed above.
%
This allows us to capture the behaviour profile of the robot's motion planner under different levels of congestion.
%
Central to our framework is a \emph{probabilistic reservation table} (PRT) which summarises robot policies as continuous-time Markov chains (CTMCs), capturing the effects of congestion.
%
To plan, we solve a sequence of single-robot MDPs, where transition probabilities and rates are obtained from the probabilistic reservation table using CTMC model checking techniques.
%
Though we focus on congestion, by considering learnt PTDs during planning we synthesise multi-robot behaviour that is robust to previously observed, but not explicitly modelled, sources of delay.
%
Empirically, we demonstrate that our framework effectively distributes robots across the environment to reduce congestion in a scalable way.
%
The congestion-aware framework is directly applicable to the CONVINCE use cases.
%
For a robotic tour guide, we can model the continuous-time behaviour of museum visitors in the PRT, and reason over human congestion during planning to take efficient routes through the museum.
%
For dual-arm assembly robots, each arm can reason over where the other is likely to be and act to avoid collisions or delays, which may occur if both arms try to place an object in the same place simultaneously.

\item[\cite{street2021congestion}] \fullcite{street2021congestion}.

Multi-robot systems must be able to maintain performance when robots get delayed during execution.
%
For mobile robots, one source of delays is \emph{congestion}.
%
Congestion occurs when robots deployed in shared physical spaces interact, as robots present in the same area simultaneously must manoeuvre to avoid each other.
%
Congestion can adversely affect navigation performance, and increase the duration of navigation actions.
%
In this paper, we present a multi-robot planning framework which utilises learnt probabilistic models of how congestion affects navigation duration. 
%
Central to our framework is a \emph{probabilistic reservation table} which summarises robot plans, capturing the effects of congestion.
%
To plan, we solve a sequence of single-robot \emph{time-varying Markov automata}, where transition probabilities and rates are obtained from the probabilistic reservation table.
% 
We also present an iterative model refinement procedure for accurately predicting execution-time robot performance.
% 
We evaluate our framework with extensive experiments on synthetic data and simulated robot behaviour.

\end{enumerate}

\end{document}