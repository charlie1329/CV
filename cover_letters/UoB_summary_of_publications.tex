\documentclass[12pt]{article}
\usepackage{fullpage}

% Publication list
\usepackage[sorting=none, backend=bibtex, doi=false, isbn=false]{biblatex}
\addbibresource{../references.bib}
\preto\fullcite{\AtNextCite{\defcounter{maxnames}{99}}}

\title{Summary of Publications}
\date{}
\author{Charlie Street}

\begin{document}
\maketitle
\thispagestyle{empty}
\begin{enumerate}
\item \fullcite{street2022context}.


Formal task-level models of multi-robot behaviour are fundamental to planning, simulation, and model checking techniques.
%
However, existing multi-robot models such as multi-agent Markov decision processes (MDPs) are invalidated by strong assumptions which fail to capture execution-time behaviour and limit their predictive power, such as simplistic duration models or synchronisation constraints.
%
In this paper we propose a novel \emph{multi-robot Markov automaton} (MA) formulation which models asynchronous multi-robot execution in continuous time.
%
MA extend MDPs by separating instantaneous decision making from stochastic action durations.
%
We capture robot dynamics, i.e. the behaviour of the controller and motion planner, using \emph{phase-type distributions} (PTDs) over action durations.
%
Further, the duration and outcome of robot actions is dependent on the precise spatiotemporal situation in which they are executed, which we refer to as the \emph{context}.
%
Contexts allow us to model the effects of environmental dynamics and robot interactions on robot action execution, where we fit a PTD for each context an action is executed in.
%
Within CONVINCE, contexts can capture human presence within a museum, which affects a robotic guide's navigation performance, or the state of a living room a robotic vacuum cleaner is operating in.
%
We also present a scalable discrete-event simulator which yields realistic statistics over execution-time robot behaviour by sampling through the multi-robot MA.
% TODO: Add avoiding computational overhead in somewhere?
%
Within CONVINCE, our simulator could be used alongside statistical model checking techniques to verify robot behaviour, or to rapidly evaluate contingency plans during execution.
%
Empirically, we demonstrate that our MA formulation can accurately predict the task-level behaviour of a multi-robot system simulated in Gazebo.

\item \fullcite{street2020multi}.

When planning for multi-robot navigation tasks under uncertainty, plans should prevent robots from colliding while still reaching their goal. 
%
Solutions achieving this fall on a spectrum. 
%
At one end are solutions which prevent robots from being in the same part of the environment simultaneously at planning time, ignoring the robots' capabilities to manoeuvre around each other, whilst at the other end are solutions that solve the problem at execution time, relying solely on online conflict resolution. 
%
Both approaches can lead to inefficient behaviour. 
%
In this paper, we present a novel framework in the middle of this spectrum that explicitly reasons over the effect the presence of multiple robots has on navigation performance. 
%
We refer to this effect as \emph{congestion}. 
%
We present a structure, called the \emph{probabilistic reservation table}, which summarises the plans of robots, allowing us to probabilistically model congestion.
%
We show how this structure can be used for planning by proposing an approach that, for each robot, sequentially builds and solves a Markov decision process where the transition probabilities are obtained by querying the probabilistic reservation table.
%
We carry out experiments on synthetic data and in simulation to show the effectiveness of our framework.

\item \fullcite{street2021congestion}.

Multi-robot systems must be able to maintain performance when robots get delayed during execution.
%
For mobile robots, one source of delays is \emph{congestion}.
%
Congestion occurs when robots deployed in shared physical spaces interact, as robots present in the same area simultaneously must manoeuvre to avoid each other.
%
Congestion can adversely affect navigation performance, and increase the duration of navigation actions.
%
In this paper, we present a multi-robot planning framework which utilises learnt probabilistic models of how congestion affects navigation duration. 
%
Central to our framework is a \emph{probabilistic reservation table} which summarises robot plans, capturing the effects of congestion.
%
To plan, we solve a sequence of single-robot \emph{time-varying Markov automata}, where transition probabilities and rates are obtained from the probabilistic reservation table.
% 
We also present an iterative model refinement procedure for accurately predicting execution-time robot performance.
% 
We evaluate our framework with extensive experiments on synthetic data and simulated robot behaviour.

\end{enumerate}

\end{document}