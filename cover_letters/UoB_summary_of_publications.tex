\documentclass[12pt]{article}
\usepackage{fullpage}

% Publication list
\usepackage[sorting=none, backend=bibtex, doi=false, isbn=false]{biblatex}
\addbibresource{../references.bib}
\preto\fullcite{\AtNextCite{\defcounter{maxnames}{99}}}

\title{\vspace{-60pt}Summary of Publications}
\date{}
\author{Charlie Street}
\pagenumbering{gobble}

\begin{document}
\maketitle
\thispagestyle{empty}
\begin{enumerate}
\item[\cite{street2022context}] \fullcite{street2022context}.


Formal task-level models of multi-robot behaviour are fundamental to planning, simulation, and model checking techniques.
%
However, existing multi-robot models such as multi-agent Markov decision processes (MDPs) are invalidated by strong assumptions which fail to capture execution-time behaviour and limit their predictive power, such as simplistic duration models or synchronisation constraints.
%
In this paper we propose a novel \emph{multi-robot Markov automaton} (MA) formulation which models asynchronous multi-robot execution in continuous time.
%
MA extend MDPs by separating instantaneous decision making from stochastic action durations.
%
We capture robot dynamics, i.e. the behaviour of the controller and motion planner, using \emph{phase-type distributions} (PTDs) over action durations.
%
Further, the duration and outcome of robot actions is dependent on the precise spatiotemporal situation in which they are executed, which we refer to as the \emph{context}.
%
Contexts allow us to model the effects of environmental dynamics and robot interactions on robot action execution, where we fit a PTD for each context an action is executed in.
%
Within CONVINCE, contexts can capture human presence within a museum, which affects a robotic tour guide's navigation performance, or the state of a living room a robotic vacuum cleaner is operating in.
%
We also present a scalable discrete-event simulator which yields realistic statistics over execution-time robot behaviour by sampling through the multi-robot MA.
% TODO: Add avoiding computational overhead in somewhere?
%
Within CONVINCE, our simulator could be used alongside statistical model checking techniques to verify robot behaviour, or to rapidly evaluate contingency plans during execution.
%
Empirically, we demonstrate that our MA formulation can accurately predict the task-level behaviour of a multi-robot system simulated in Gazebo.

\item[\cite{street2020multi}] \fullcite{street2020multi}.

Multi-robot systems must be able to maintain performance when robots get delayed during execution.
%
For mobile robots, one source of delays is congestion.
%
Congestion occurs when robots deployed in shared physical spaces interact, as robots present in the same area simultaneously must manoeuvre to avoid each other.
%
Congestion can adversely affect navigation performance, and increase the duration of navigation actions.
%
In this paper, we present a multi-robot planning framework which utilises learnt probabilistic models of how congestion affects navigation duration.
%
We represent these duration models as PTDs, similar to the multi-robot MA described above.
%
Central to our framework is a \emph{probabilistic reservation table} (PRT), which captures the effects of congestion by summarising robot policies as \emph{continuous-time Markov chains} (CTMCs).
%
To plan, we solve a sequence of single-robot MDPs, where transition probabilities and rates are obtained from the probabilistic reservation table using CTMC model checking techniques.
%
Though we focus on congestion, by reasoning over learnt duration models during planning we synthesise multi-robot behaviour that is robust to previously observed, but not explicitly modelled, sources of delay.
%
Empirically, we demonstrate that our framework effectively distributes robots across the environment to reduce congestion in a scalable way.
%
Our congestion-aware planning framework is directly applicable to the CONVINCE use cases.
%
For a robotic tour guide, we can model the continuous-time behaviour of museum visitors in the PRT, and reason over human congestion during planning to take efficient routes through the museum.
%
For dual-arm assembly robots, each arm can reason over where the other is likely to be to avoid collisions or delays, which occur when both arms try to place an object in the same place simultaneously.\looseness=-1

\item[\cite{street2021congestion}] \fullcite{street2021congestion}.

This paper extends the congestion-aware planning framework in~\cite{street2020multi}.
%
We introduce \emph{time-varying Markov automata} (TVMA), a novel formalism for continuous-time planning problems under time-varying dynamics.
%
We use TVMAs to model the single-robot planning problem under congestion, which we solve sequentially for each robot as in~\cite{street2020multi}.
%
TVMAs could be used in CONVINCE to model how a robotic vacuum cleaner is less likely to clean successfully during times of the day when people are often present, for example.
%
By solving the TVMA, the vacuum cleaner would then plan to clean when rooms are likely to be empty.
%
We present an approximate TVMA solution method which extends labelled real-time dynamic programming to incrementally build and solve an MDP abstraction of the TVMA.
%
To build this abstraction, we discretise continuous duration distributions and include time in the MDP state.
%
As well as synthesising robot behaviour, we also wish to accurately predict the execution-time performance of the multi-robot system.
%
To predict performance, we require accurate models of robot policy execution under congestion.
%
After planning, the PRT stores a CTMC for each robot that describes the continuous-time behaviour induced by their policy.
%
However, due to the sequential nature of our planning framework, these CTMCs contain inaccurate congestion information.
%
Therefore, we present an iterative procedure that refines the single-robot CTMCs stored in the PRT to accurately capture the effects of the other robots. 
%
We then predict execution-time robot performance by using formal verification techniques to obtain probabilistic guarantees on the refined CTMCs.
%
These techniques could be used in CONVINCE to identify unexpected robot failures or delays during execution, and trigger replanning when necessary.
%
Alternatively, by predicting robot behaviour prior to execution, we can identify when and where contingency plans are likely to be required.\looseness=-1


\end{enumerate}

\end{document}