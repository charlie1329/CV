\documentclass[11pt]{article}
\usepackage{fullpage}
\usepackage{multicol}
\usepackage[top=1in,bottom=1in]{geometry}
\usepackage[hidelinks]{hyperref}
\usepackage[normalem]{ulem}

% Publication list
\usepackage[sorting=none, backend=bibtex, doi=false, isbn=false]{biblatex}
\addbibresource{references.bib}
\preto\fullcite{\AtNextCite{\defcounter{maxnames}{99}}}

\title{\vspace{-70pt}\Huge\underline{Charlie Street}}
\date{}

\begin{document}
\pagenumbering{gobble}
\maketitle
\vspace*{-60pt}

% Personal Info
\begin{flushleft}
\noindent
\large 17 Temple Street,
\hfill 
\large +44 7917601977 \\
\large Oxford. 
\hfill
\large
\large \href{mailto:me@charliestreet.net}{\url{me@charliestreet.net}} \\
\large OX4 1JS.
\hfill
\large
\url{https://ori.ox.ac.uk/people/charlie-street} \\
\end{flushleft}
	
\section*{\uline{Research}}	
        
I am a DPhil student in the Goal-Oriented Autonomous Long-Lived Systems (GOALS) Lab at the Oxford Robotics Institute, University of Oxford.
%
My research is focused on the robust continuous-time coordination of multi-robot systems under uncertainty.
%
To achieve this, I apply planning, model checking, and task allocation techniques to continuous-time models of multi-robot behaviour.\\

\noindent \textbf{Research Interests}
\begin{itemize}
    \item Multi-Robot Planning Under Uncertainty
    \item Formal Methods for Multi-Robot Systems
    \item Continuous-Time and Non-Stationary Planning Models
    \item Multi-Robot Task Allocation
\end{itemize}
	
\section*{\uline{Education}}
\begin {itemize}
\item \textbf{DPhil in Engineering Science at the University of Oxford \hfill 2018-2022} 
\begin{itemize}
    \item Thesis: \emph{Multi-Robot Coordination Under Temporal Uncertainty}
    \item Supervisors: Nick Hawes, Bruno Lacerda, and Manuel M{\"u}hlig
    \item Expected Submission Date: June 2022
\end{itemize}
			
\item \textbf{MSci in Computer Science at the University of Birmingham \hfill 2014-2018}
\begin{itemize}
\item Thesis: \emph{IntelliJam: An Intelligent Agent for Musical Improvisation}
\item Supervisor: Peter Tino
\item Degree Class: First Class with Honours (Average: 92\%)
\item Awarded Undergraduate Distinguished Dissertation Prize 2018
\item Awarded Best in Degree Programme 2014/15, 2015/16, 2016/17, and 2017/18
\item Awarded IBM Team Project Prize 2015/16
\item Awarded BCS Prize for Best in Year 2014/15
\end{itemize}
\end{itemize}

\section*{\underline{Projects}}
\begin{itemize}
\item \textbf{First Fleet\hfill 2020-2021}
\begin{itemize}
\item Deploying Multi-Robot Systems in Agricultural Environments
\item Implemented Multi-Robot Planning System
\end{itemize}
\item \textbf{Team ORIon (RoboCup Competition Team)\hfill 2019-2021}
\begin{itemize}
\item Deploying Service Robots in Domestic Environments
\item Led Team ORIon and Task-Level Planning Sub-Team
\end{itemize}
\end{itemize}

\section*{\underline{Supervision}}
\noindent \textbf{Fourth Year Projects}
\begin{itemize}
\item \textbf{Alex Rutherford (with Bruno Lacerda and Nick Hawes)\hfill 2021 - 2022}
\begin{itemize}
\item Topic: \emph{Multi-Agent Reinforcement Learning with a Model-Based Simulator}
\end{itemize}
\item \textbf{Yifeng Wei (with Bruno Lacerda) \hfill 2020 - 2021}
\begin{itemize}
\item Topic: \emph{Trial-Based Search for Generalised Stochastic Petri Nets}
\end{itemize} 
\item \textbf{James Wheadon (with Nick Hawes) \hfill 2019 - 2020}
\begin{itemize}
\item  Topic: \emph{Multi-Agent Path Finding in Continuous Time}
\end{itemize}
\item \textbf{Han Zhou (with Bruno Lacerda) \hfill 2018 - 2019}
\begin{itemize}
\item Topic: \emph{Auctioning for Multi-Robot Coordination}
\end{itemize}
\end{itemize}

\noindent \textbf{Internships}	
\begin{itemize}
\item \textbf{Tom Liu (with Nick Hawes)\hfill 2021}
\begin{itemize}
\item Topic: \emph{Generalising Duration Distributions across Topological Maps}
\end{itemize}
\item \textbf{Clarissa Costen (with Nick Hawes)\hfill 2019}
\begin{itemize}
\item Topic: \emph{Continuous-Time Markov Chains for Shared Autonomy}
\end{itemize}
\end{itemize}

\section*{\underline{Outreach}}
\begin{itemize}
\item \textbf{Led Robot Demonstrations at Goodwood Festival of Speed\hfill 2021}
\item \textbf{Led Robot Demonstration at University Open Day\hfill 2019}
\item \textbf{Assisted with Robot Demonstration at Blenheim Palace\hfill 2019}
\end{itemize}

\iffalse
	\section*{\underline{Technical Skills}}
		\subsection*{\underline{Programming Languages}}
		\renewcommand{\arraystretch}{1.3}%
		\begin{tabular}[20pt]{ll}
		        \textbf{Python} & I am very familiar with Python, having used it almost exclusively since starting my DPhil. \\
			\textbf{C} & I have strong experience with memory management, pointers etc. \\
			\textbf{C++} & I can use classes and templates on top of the underlying C functionality. \\
			\textbf{Haskell} & I have a reasonable understanding of Haskell and the functional paradigm.\\
			\textbf{Agda} & I can formulate inductive proofs over basic numbering systems and data structures.\\
			 \textbf{Java} &  I have a strong level of proficiency, having used Java heavily during my time at university.\\
			 \textbf{OCaml} & I have an understanding of the syntax and underlying concepts of the language.\\
		\end{tabular} 
		\subsection*{\underline{Other}}
		\renewcommand{\arraystretch}{1.3}%
		\begin{tabular}{ll}
			 \textbf{Git} & I have experience using Git, having used it for any significant project I have partaken in.\\
			\textbf{LaTeX} & I've produced many documents in LaTeX, notably my dissertation.\\
			\textbf{ROS} & I've had experience working with/running robotics systems using the ROS middleware.
		\end{tabular}
\fi
			
\section*{\underline{Publications}}
\begin{itemize}
\item[\cite{street2022context}] \fullcite{street2022context}.
\item[\cite{street2021congestion}] \fullcite{street2021congestion}.	\item[\cite{street2020multi}] \fullcite{street2020multi}.
\end{itemize}


\end{document}