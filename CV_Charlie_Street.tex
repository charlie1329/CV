\documentclass[11pt]{article}
\usepackage[margin=65pt]{geometry}
\usepackage{multicol}
\usepackage[hidelinks]{hyperref}
\title{\vspace{-80pt}\underline{Charlie Street}}
\date{}
\hyphenation{Programme}%lets me remove hyphenation for specific words
\hyphenation{Computer}
\hyphenation{Programming}
\hyphenation{Computing}
\hyphenation{Operating}
\hyphenation{Birmingham}
\hyphenation{Waterloo}
\hyphenation{IntelliJam}
\begin{document}
	\pagenumbering{gobble}
	\maketitle
	\vspace*{-60pt}
	%\begin{flushleft}
	%	\noindent
	%	\large7 Cassiobury Road, Weymouth, Dorset. DT4 7JN\\                         %MAKE THIS LOOK NICER     
	%	\large 07917601977\\
     %     		 \large me@charliestreet.net
	%\end{flushleft}
	\begin{flushleft}
	\noindent
	\large 17 Temple Street,
	\hfill
	\large +44 7917601977 \\
	\large Oxford,
	\hfill 
	\large \href{mailto:me@charliestreet.net}{\nolinkurl{me@charliestreet.net}} \\
	\large Oxfordshire. 
	\hfill
	\large
	\url{https://github.com/charlie1329/}\\
	\large OX4 1JS.
	\hfill
	\large
	\url{https://ori.ox.ac.uk/ori-people/charlie-street/} \\
	\end{flushleft}

	\vspace{-10pt}
	\begin{center}
			\textit{\large As a highly driven individual with a strong academic record, I throw myself into any challenge I am confronted with. With extra-curricular experience working in teams, I believe I can add value to any team-based project.}
	\end{center}
			
	\vspace{-30pt}
	\hspace{-100pt}\section*{\underline{Education}}
		\begin {itemize}
			\item \textbf{Postgraduate (2018-2022)} Studying DPhil Engineering Science at the University of Oxford.
				\begin{itemize}
					\item Thesis Title: Multi-Robot Coordination under Temporal Uncertainty.
					\item Supervised by Prof. Nick Hawes (Oxford Robotics Institute), Dr. Bruno Lacerda (Oxford Robotics Institute) \& Dr.-Ing. Manuel M{\"u}hlig (Honda Research Institute).
					\item My research interests are:
					    \begin{itemize}
					        \item Multi-Robot Systems
					        \item Planning Under Uncertainty
					        \item Continuous-Time Planning Models
					        \item Formal Verification
					    \end{itemize}
				\end{itemize}
		
			\item \textbf{Undergraduate (2014-2018)} Studied MSci Computer Science at the University of Birmingham.
				\begin{itemize}
					\item Achieved average grades of \textbf{94\%}, \textbf{95\%}, \textbf{89.5\%} and \textbf{93.3\%} in each year chronologically.
					\item Graduated July 2018 with a First Class with Honours (overall degree average of \textbf{92\%}).
					\item Won Best in Degree Programme 2014/15, 2015/16, 2016/17 \& 2017/18.
					\item Won Undergraduate Distinguished Dissertation Prize 2018.
					\item Won BCS Prize for Best in Year 2014/15.
					\item Won IBM Team Project Prize 2015/16.
					\item Awarded School of Computer Science Excellence Scholarship during first year.
					\item In April $2018$ I submitted my master's dissertation titled: `IntelliJam: An Intelligent Agent for Musical Improvisation'.
					\item Completed modules in Intelligent Robotics, Machine Learning, Neural Computation, Operating Systems, Networks, Functional Programming and Computer Security among others.
				\end{itemize}
			
		\end{itemize}
		
        \vspace{-20pt}
	\hspace{-100pt}\section*{\underline{Publications}}
	\begin{itemize}
	\item Charlie Street, Bruno Lacerda, Manuel M{\"u}hlig, and Nick Hawes. "Multi-Robot Planning Under Uncertainty with Congestion-Aware Models". In: \textit{Proc. of the $19$th International Conference on Autonomous Agents and Multi-Agent Systems (AAMAS)}. Auckland, New Zealand, $2020$.
	\end{itemize}
	
	\vspace{-20pt}
	\hspace{-100pt}\section*{\underline{Student Supervision}}
	\begin{itemize}
	\item \textbf{4th Year Project Supervisor:} Han Zhou (October $2018$ - May $2019$)
	    \begin{itemize}
	        \item Topic: Auctioning for Multi-Robot Coordination 
	    \end{itemize}
	\item \textbf{4th Year Project Supervisor:} James Wheadon (October $2019$ - Present)
	    \begin{itemize}
	        \item Topic: Multi-Agent Path Finding in Continuous-Time
	    \end{itemize}
	\end{itemize}
	
	\vspace{-20pt}
	\hspace{-100pt}\section*{\underline{Technical Skills}}
		\subsection*{\underline{Programming Languages}}
		\renewcommand{\arraystretch}{1.3}%
		\begin{tabular}[20pt]{ll}
		        \textbf{Python} & I am very familiar with Python, having used it almost exclusively since starting my DPhil. \\
			\textbf{C} & I have strong experience with memory management, pointers etc. \\
			\textbf{C++} & I can use classes and templates on top of the underlying C functionality. \\
			\textbf{Haskell} & I have a reasonable understanding of Haskell and the functional paradigm.\\
			\textbf{Agda} & I can formulate inductive proofs over basic numbering systems and data structures.\\
			 \textbf{Java} &  I have a strong level of proficiency, having used Java heavily during my time at university.\\
			 \textbf{OCaml} & I have an understanding of the syntax and underlying concepts of the language.\\
		\end{tabular} 
		\subsection*{\underline{Other}}
		\renewcommand{\arraystretch}{1.3}%
		\begin{tabular}{ll}
			 \textbf{Git} & I have experience using Git, having used it for any significant project I have partaken in.\\
			\textbf{LaTeX} & I've produced many documents in LaTeX, notably my dissertation.\\
			\textbf{ROS} & I've had experience working with/running robotics systems using the ROS middleware.
		\end{tabular}
	
	\vspace{-10pt}
	\hspace{-100pt}\section*{\underline{Projects}} %FINISH THIS AND CLEAN UP
			\begin{itemize}
			        \item \textbf{RoboCup (2018-Present)} I am the leader of team ORIon - the University of Oxford's RoboCup team. The aim is to develop a general purpose service robot for domestic tasks. My work in the team is concerned with high-level task planning. We have previously competed in the RoboCup@Home DSPL $2019$ in Sydney, where we placed $6$th, and have more competitions planned throughout $2020$. The team has also previously demonstrated for HRH Prince William. 
				\item \textbf{IntelliJam (2017-2018)} The software created alongside my master's dissertation. IntelliJam uses Fractal Prediction Machines to allow a guitar player to play whilst connected to a computer and have an agent respond to their playing with new musical phrases in real time. In addition to the agent, a new method of melody extraction based on spectrogram filtering was devised.
				\item \textbf{Dating Chat-Bot for `The Gadget Show' (2016)} A project for the TV show to create a bot to partake in speed-dating. The goal was for an unknowing subject to choose the bot over a human. I led the back-end/AI sub-team; I was the majority contributor to the design and implementation of the bot. This forced me to think creatively while under time-pressure.
				\item \textbf{Simulizer (2016-2017)} A project initially undertaken for a second year module but continued since. Simulizer is a simulator and visualiser for a MIPS R3000 processor. Working in a team of 5, I was in charge of the back-end, requiring me to write a faithful simulation of the R3000 processor, including a primitive pipeline for execution. This has since been used as a significant teaching aid for the `Computer Systems \& Architecture' module at the University of Birmingham. Students had to write assignments in Simulizer; the assignments were graded using the software. 
				\item See \textbf{\href{https://github.com/charlie1329/}{GitHub}} for more projects.
			\end{itemize}
        \iffalse
	\vspace{-20pt}
	\hspace{-100pt}\section*{\underline{Extra-Curricular Interests And Experiences}}
			\begin{itemize}
				\item \textbf{Music} I am an avid musician, having played the guitar since the age of 12 with interest in many musical genres. This has required great commitment and perseverance. Previously, I have been lead guitarist/backing vocalist for a rock covers band. Being an active member of a band has improved my ability to cooperate well within a group and listen to those around me. 
				\item \textbf{Duke Of Edinburgh Award} I've completed all 3 levels of the DofE award scheme. This required large amounts of team-work and communication, particularly during expeditions, which involved working in a team of 4 for 4 days on Dartmoor. This greatly benefit my leadership skills. I also partook in voluntary work including working in a local charity shop and aiding restoration work along a canal near Gloucester.
			\end{itemize}

				
	\vspace{-20pt}
	\hspace{-100pt}\section*{\underline{References}}	
				\begin{multicols}{2}
					\noindent
					Prof. Nick Hawes:  \href{mailto:nickh@robots.ox.ac.uk}{\nolinkurl{nickh@robots.ox.ac.uk}}\\
					Oxford Robotics Institute,\\
					University of Oxford,\\
					Oxford, \\
					Oxfordshire.\\
					OX2 6NN.\\
	
					\noindent
					Prof. Peter Tino:  \href{mailto:P.Tino@cs.bham.ac.uk}{\nolinkurl{P.Tino@cs.bham.ac.uk}}\\
					School of Computer Science,\\
					University of Birmingham,\\
					Birmingham, \\
					West Midlands.\\
					B15 2TT.\\
				\end{multicols}
        \fi


\end{document}